%
%

\chapter{Der abstrakte Syntaxgraph}
\label{syngraph}

\section{Grundlagen}

Nachdem der Parser seine Arbeit getan hat, findet sich der VHDL Text
m�glichst unverf�lscht in einem Syntaxgraphen wieder. Die Knoten dieses
Graphen enthalten dabei einige Attribute, z.B. den Namen einer
Variablen als "char*" oder die Zeilennummer eines Konstrukts als "int".
Der Rest steckt in den Verbindungen der Knoten untereinander, z.B.
werden die Operanden "a" und "b" des Ausdrucks "a+b" als eigene Knoten
dargestellt, und der Knoten f�r den gesamten Ausdruck hat Verbindungen
zu ihnen. 'Graphisch' sieht das in diesem Buch so aus:

\npic{intro}

Auf diese Weise werden auch alle weiteren Bilder von (Teil-) Graphen
dargestellt: Jeder Knoten ist ein Kasten mit den Attributen und Namen
f�r die Verbindungen; Verbindungen sind Pfeile von einem Kasten zum
anderen.  Der hervorgehobene Name in der ersten Zeile eines Kastens
bezeichnet den Typen des zugeh�rigen Knotens, z.B. "Declaration",
"Statement" oder "Expression".  Die Typen sind alle direkt oder indirekt
von einem Basistypen abgeleitet und bilden so einen 'Typenbaum'.
\footnote{Mehrfachvererbung gibt es leider (noch) nicht.}

Realisiert ist die Hierarchie der Knotentypen mit einer passenden
\C++ Klassenhierarchie. Die Verbindungen zwischen den Knoten werden dabei
durch Zeiger hergestellt. Die Erzeugung dieser Klassenhierarchie aus
einer extra spezial Knotenbeschreibungssprache ist ein Kapitel f�r
sich.\footnote{Vielleicht sogar ein ganzes Buch} "ctree" ist
haupts�chlich f�r zwei Dinge zust�ndig, die \C++ (noch) nicht bietet:
eine simple Typenidentifizierung zur Laufzeit und eine automatisierte
M�glichkeit, den Graphen oder Teile davon in eine Datei zu schreiben
und wieder zu lesen. ??? enth�lt alle Details, hier wird
nur kurz beschrieben, wie seine M�glichkeiten f�r "libvaul" genutzt
worden sind.

Knotentypen werden in einer eigenen Notation definiert. Z.B. k�nnte
die Definitionen f�r die Knotentypen "BinOp" und "VariableRef" oben
aus dem Bild so aussehen:
\begin{code}
// {\rm Eine allgemeine Basisklasse f�r Ausdr�cke.}
// {\rm (Node ist die Basisklasse der kompletten Typenhierarchie und}
// {\rm mu� voanders definiert sein.)}

node Expression : Node \{
\};

// {\rm Ein \verb|BinOp| ist auch eine \verb|Expression| mit Verbindungen}
// {\rm (Zeigern) zu den beiden Operanden.}

node BinOp : Expression \{
    attr char op;
    Expression left;
    Expression right;
\};

// {\rm Eine \verb|VariableRef| hat keine Zeiger, ist also ein}
// {\rm "Blatt" des Graphen.}

node VariableRef : Expression \{
    attr char *id;
\};
\end{code}

"ctree" erzeugt daraus \C++ Klassen und Typdefinitionen nach folgenden
Konventionen, am Beispiel von "BinOp" aufgez�hlt:

\begin{desc}{\b{sBinOp}}
Der Name der eigentlichen Klasse:
\begin{code}
struct sBinOp \{
    {\rm \dots}
    char op;
    nExpression left;
    nExpression right;
    {\rm \dots}
\};
\end{code}
\end{desc}

\begin{desc}{\b{nBinOp}}
Ein Name f�r den Typ 'Zeiger auf "sBinOp"':
\begin{code}
typedef sBinOp *nBinOp;
\end{code}
\end{desc}

\begin{desc}{\b{nkBinOp}}
Eine Enumerationskonstante vom Type "nKind" �ber die sich
der Typ eines Knotens identifizieren l��t:
\begin{code}
enum nKind {
    {\rm \dots}
    nkBinOp,
    {\rm \dots}
};
\end{code}
"ctree" listet die Konstanten in der Enumeration in der Reihenfolge
auf, die ihm am Besten passt.
\end{desc}

\begin{desc}{\b{mBinOp}}
Ein Macro zum Erzeugen von "BinOp" Knoten mit
"new". Nicht benutzen bevor es nicht besser dokumentiert ist.
\end{desc}

Die Basisklasse der Typenhierachie (und damit alle abgeleiteten
Klassen) hat eine Funktion zum �berpr�fen des echten Typs zur
Laufzeit:

\begin{desc}{"bool" \b{is} ("nKind" <k>)}
Ermittelt ob "this" zu dem Typ oder einem davon abgeleiteten geh�rt,
der durch <k> identifiziert wird.
\end{desc}

Ein Beispiel:
\begin{code}
// {\rm gibt \verb|e| in Postfix aus}

void postfix(nExpression e)
\{
    if(e->is(nkBinOp)) \{
        nBinOp bo = nBinOp(e);
        postfix(bo->left);
        postfix(bo->right);
        printf("%c ", bo->op);
    \} else if(e->is(nkVariableRef))
        printf("%s ", nVariableRef(e)->id);
    else
        printf("? ");
\}
\end{code}

Die Beschreibung der einzelnen Knotentypen folgt diesem Muster:

\def\internal{{\sl (intern)}}
\begin{ndesc}{\b{Typname} : Basis : Basis_der_Basis : ... : Node}{\ \internal}
 Zuerst kommt der 'Stammbaum' des Typs: Ausgehend von der diskutierten
 Klasse werden alle Basistypen bis hin zur Wurzelklasse aufgelistet.
 Knotentypen, die mit ``\internal'' markiert sind, werden vom Parser
 nur zum eigenen Vergn�gen benutzt. Sie tauchen nicht im endg�ltigen
 Syntaxgraphen auf, der in die Bibliothek eingef�gt wird.

 In diesem Absatz folgt dann eine allgemeine Einordnung in die
 Hierarchie und die Rolle dieses Knotentypen bei der Darstellung von
 VHDL Syntax.

 \begin{desc}{"c_type" \b{attribut_name}}
  Beschreibt die Attribute des Typs. Die Typen auf der linken Seite
  sind normale \C++ Typen.
 \end{desc}

 \begin{desc}{Knotentyp \b{kind}}
  Beschreibt Verbindungen zu anderen Knoten. In der \C++ Implementierung
  sind das Zeiger auf die entsprechenden Strukturen ("nKnotentyp", s.o.).
 \end{desc}

 Es werden nur die Unterschiede zum direkten Basistypen beschrieben. Das
 beinhaltet alle neuen Attribute und Verbindungen des abgeleiteten Typs
 und spezielle Interpretationen und Beschr�nkungen f�r ererbte Elemente.
\end{ndesc}

Neben den Knotentypen werden auch einige \C++-Typen verwendet, die
mittels "typedef" passende Namen bekommen.

Knotentypen, die mit ``\internal'' markiert sind, werden vom
Parser nur zum eigenen Vergn�gen benutzt. Sie tauchen nicht im
endg�ltigen Syntaxgraphen auf, der in die Bibliothek eingef�gt wird.

\paragraph{Listen}

Einfach verkette Listen von Knoten tauchen sehr h�ufig im Graphen auf.
Der Zeiger auf den ersten Knoten einer solchen Liste hei�t immer
"first" oder "first_"*. Die einzelnen Elemente werden entweder �ber
"next"- oder "link"-Zeiger zusammengeh�ngt. Bei Listen, bei denen die
Reihenfolge der Elemente eine Rolle spielt (z.B. die Liste der Indices
eines Arrays) wird "next" verwendet, bei ungeordneten Listen
(z.B. ???)  hei�t der Zeiger auf das n�chte Element "link". Das Ende
einer Liste wird durch einen "NULL"-Zeiger als "next" oder "link"
gekennzeichnet. Wenn ein Knoten gleichzeitig in mehreren Listen
auftauchen kann, haben die Zeiger auf den n�chsten Knoten die Namen
"next_"* oder "link_"*.

Soweit nicht anders erw�hnt, sind die Elemente von geordneten Listen
in der gleichen Reihenfolge aufgelistet wie im VHDL-Text.

 \section{Basis der Typenhierarchie} 
\smallskip
\begin{ndesc}{\b{Node}}{}
 Alle anderen Knotentypen sind von <Node> abgeleitet; einer mu� ja
  mal den Anfang machen.

 \begin{desc}{"vaul_design_unit *" \b{owner};}
   Ordnet jedem Knoten die "vaul_design_unit" zu, aus dessen Analyse er
   hervorgegangen ist. Knoten, die von einer "vaul_library" oder einem
   "vaul_lib\-pool" kommen, haben garantiert $<owner> \neq "NULL"$, alle
   anderen k�nnen auch $<owner> = "NULL"$ haben.
  \end{desc}

\end{ndesc}

\smallskip
\begin{ndesc}{\b{PosNode} : Node}{}
 Ausgew�hlte Knotentypen (fast alle) haben zus�tzlich zum <owner> noch
  eine Angabe �ber die Zeilennummer des VHDL Textes, der zu diesem
  Knoten gef�hrt hat. Sehr n�tzlich f�r Fehlermeldungen.

 \shortdesc{"int" \b{lineno};}
\end{ndesc}

 \section{Namen}

\begin{ndesc}{"typedef char *"\b{Id};}{}
Alle Attribute die einen einfachen Namen bedeuten, haben diesen Typ.
\end{ndesc}

\smallskip
\begin{ndesc}{\b{IdList} : PosNode : Node}{}
 <IdList>s sind eine einfache Auflistung von <Id>s. Die Reihenfolge
  der <Id>s in der Liste entspricht nicht unbeding der Reihenfolge im
  VHDL Text. 
 \shortdesc{"Id" \b{id};}
 \shortdesc{IdList \b{link};}
\end{ndesc}

\smallskip
\begin{ndesc}{\b{Name} : PosNode : Node}{\ \internal}
 Die Basis f�r alle Namenknoten.
\end{ndesc}

\smallskip
\begin{ndesc}{\b{SimpleName} : Name : PosNode : Node}{\ \internal}
 <SimpleName>s sind einfache Namen, die nur aus einem "Id" bestehen.
 \shortdesc{"Id" \b{id};}
\end{ndesc}

\smallskip
\begin{ndesc}{\b{OpSymbol} : Name : PosNode : Node}{\ \internal}
 Ein <OpSymbol> beschreibt ein Operator Symbol.
 \begin{desc}{"Id" \b{op};}
Der Name des Operators, komplett mit `"\dq"',
                 z.B. `"\dq and\dq"'. 
 \end{desc}
\end{ndesc}

\smallskip
\begin{ndesc}{\b{SelName} : Name : PosNode : Node}{\ \internal}
 <SelName>s stellen <selected names> der Form <prefix>.<suffix> dar.
 \begin{desc}{Name \b{prefix};}
Entweder ein weiterer <SelName> oder ein <SimpleName>.
 \end{desc}
 \shortdesc{"Id" \b{suffix};}
\end{ndesc}

\smallskip
\begin{ndesc}{\b{IftsName} : Name : PosNode : Node}{\ \internal}
 <IftsName>s erf�llen viele Aufgabe (dank der wunderbaren
  Eindeutigkeit der VHDL Grammatik). Sie stellen Indizierungen,
  Funktionsaufrufe, Typumwandlungen oder <slices> dar.

 \shortdesc{Name \b{prefix};}
 \begin{desc}{GenAssocElem \b{assoc};}
 Der jeweilige Kontext des <IftsName> bestimmt die Bedeutung
          und G�ltigkeit der verschiedenen <GenAssocElem>s in diese Liste. 
 \end{desc}
\end{ndesc}

\smallskip
\begin{ndesc}{\b{AttributeName} : Name : PosNode : Node}{\ \internal}

 \shortdesc{Name \b{prefix};}
 \shortdesc{"Id" \b{attribute};}
 \begin{desc}{NamedAssocElem \b{first_actual};}
 Die optionalen Parameter des Attributes. `Kein Parameter'
          wird durch <first_actual>$ == "NULL"$ ausgedr�ckt. 
 \end{desc}
\end{ndesc}

\smallskip
\begin{ndesc}{\b{SelNameList} : PosNode : Node}{\ \internal}

 \shortdesc{SelName \b{name};}
 \shortdesc{SelNameList \b{link};}
\end{ndesc}

 \section{Deklarationen und G�ltigkeitsbereiche}

Deklarationen geben den Dingen ihren Namen; aber auch die anonymen
Konstrukte werden als Deklarationen dargestellt. 


\smallskip
\begin{ndesc}{\b{Declaration} : PosNode : Node}{}
 Die Basis f�r alle Deklarationen. Konkrete Dinge werden durch abgeleitete
  Knotentypen dargetstellt. 
 \begin{desc}{Declaration \b{next_decl};}
 Alle <Declaration>s eines <Scope>s werden
	  mit <nect_decl> zu einer "NULL"-terminierten Liste zusammengeh�ngt.
	
 \end{desc}
 \begin{desc}{"Id" \b{id};}
 Der Name dieses Dings. Unbenannte Dinge haben $<id> == "NULL"$.
	
 \end{desc}
 \begin{desc}{Scope \b{scope};}
 <scope> zeigt auf den <Scope> der diese <Declaration> enth�lt.
	  Einige <Declaration>s sind keinem <Scope> zugeordnet und haben
	  $<scope> == "NULL"$.
	
 \end{desc}
 \begin{desc}{Declaration \b{next}();}
	  Ein G�ltigkeitsbereich kann durch mehrere <Scope>s
          dargestellt werden.\footnote{Z.B. bilden ein "package" und
          der zugeh�rige "package body" einen einzigen
          G�ltigkeitsbereich. Jeder wird aber einzeln durch einen
          <Scope>-Knoten repres�ntiert. Diese beiden <Scope>s sind
          �ber ihren <continued> Zeiger zusammengeh�ngt.}  <next()>
          wandert durch alle <Declaration>s dieser zusammenh�ngenden
          <Scope>s, w�hrend �ber <next_decl> nur die <Declaration>s
          eines einzigen <Scope>s erreichbar sind.
	  \end{desc}
	
\end{ndesc}

\smallskip
\begin{ndesc}{\b{AttributedDeclaration} : Declaration : PosNode : Node}{}
 Die Basis f�r alle Deklarationen, die mit benutzerdefinierten
  Attributen garniert werden k�nnen. Noch nicht realisiert.

 \shortdesc{AttributeValue \b{first_attribute};}
\end{ndesc}

\smallskip
\begin{ndesc}{\b{Attribute} : Declaration : PosNode : Node}{}
 Ein benutzerdefiniertes Attribut. 
 \shortdesc{Type \b{type};}
\end{ndesc}

\smallskip
\begin{ndesc}{\b{AttributeValue} : PosNode : Node}{}
 Der Wert eines benutzerdefinierten Attributes aus einer
  <attribute specification>. Noch nicht realisiert.

 \shortdesc{AttributeValue \b{next};}
 \shortdesc{Attribute \b{attribute};}
 \shortdesc{Expr \b{value};}
\end{ndesc}

\smallskip
\begin{ndesc}{\b{Scope} : AttributedDeclaration : Declaration : PosNode : Node}{}
 Ein <Scope> enth�lt alle <Declaration>s einer <declarative region>.

 \begin{desc}{Scope \b{continued};}
 Alle <Scope>s eines G�ltigkeitsbereichs werden �ber <continued>
	  zu einer "NULL"-terminierten Liste zusammengeh�ngt.
	
 \end{desc}
 \begin{desc}{Declaration \b{first_decl};}
 Die erste <Declaration> dieser <declarative region>.
 \end{desc}
\begin{desc}{Declaration \b{first}();}
	 Liefert die erste <Declaration> des G�ltigkeitsbereichs, der mit
	 diesem <Scope> beginnt.
	 \end{desc}
	
\end{ndesc}

\smallskip
\begin{ndesc}{\b{TopScope} : Scope : AttributedDeclaration : Declaration : PosNode : \dots}{}
 Ein <TopScope> repr�sentiert die Umgebung einer <design unit> und nimmt
 die Informationen aus <library> und <use clauses> auf sowie die <design unit>
 selbst. Ein <TopScope> ist der einzige <Scope> mit $<scope> == "NULL"$

\end{ndesc}

\smallskip
\begin{ndesc}{\b{LibNameDecl} : Declaration : PosNode : Node}{}
 <LibNameDecl>s tauchen nur in einem <TopScope> auf und nehmen die
  Informationen aus <library clauses> auf.

\end{ndesc}

\smallskip
\begin{ndesc}{\b{IndirectDecl} : Declaration : PosNode : Node}{}
 <use clauses> werden in <IndirectDecl>s umgesetzt. 

 \begin{desc}{Package \b{ind_scope};}
 Der <Scope>, dessen Deklarationen im <Scope> dieser <IndirectDecl>
          potentiell sichtbar sein sollen. $<id> == "NULL"$ bedeutet, da�
          alle Deklarationen in <ind_scope> betrachtet werden sollen,
          ansonsten sind nur die potentiell sichtbar, dessen <id> mit der
          <id> dieser <IndirectDecl> �bereinstimmen.
	
 \end{desc}
\end{ndesc}

 \subsection{Design Units}

<design units> werden allesamt durch Knoten dargestellt, die von
<Scope> abgeleitet sind. Diese Knoten k�nnen nur innerhalb eines
<TopScopes> auftauchen.

\smallskip
\begin{ndesc}{\b{Package} : Scope : AttributedDeclaration : Declaration : PosNode : \dots}{}
 \ 
\end{ndesc}

\smallskip
\begin{ndesc}{\b{StandardPackage} : Package : Scope : AttributedDeclaration : \dots}{}
 Da die Deklarationen aus dem <package> "std.standard" f�r viele Teile des
 Compilers wichtig sind, werden die ben�tigten Deklarationen hier notiert.

 \shortdesc{Type \b{predef_BIT};}
 \shortdesc{Type \b{predef_BOOLEAN};}
 \shortdesc{Type \b{predef_INTEGER};}
 \shortdesc{Type \b{predef_REAL};}
 \shortdesc{Type \b{predef_TIME};}
 \shortdesc{Type \b{predef_STRING};}
 \shortdesc{Type \b{predef_BIT_VECTOR};}
 \shortdesc{Type \b{predef_SEVERITY_LEVEL};}
 \shortdesc{Type \b{predef_FILE_OPEN_KIND};}
 \shortdesc{UniversalInteger \b{universal_integer};}
 \shortdesc{UniversalReal \b{universal_real};}
\end{ndesc}

\smallskip
\begin{ndesc}{\b{PackageBody} : Scope : AttributedDeclaration : Declaration : \dots}{}
 <continued> zeigt auf das zugeh�rige <Package>. F�r Konstanten, die
  im <Package> nicht initialisiert wurden, kann dies im <PackageBody> mit
  einem <ConstantBody> nachgeholt werden.

\end{ndesc}

 \subsection{Typen} 

Die Knotentypen der Typen and Subtypen bilden eine gemeinsame Hierarchie mit
der Basis <Type>, obwohl ein VHDL Subtype kein Type ist. Dadurch wird es
einfacher gleichzeitig mit Typen oder Subtypen zu arbeiten.

Typen (und Subtypen) sind <Declaration>s. Viele dieser Deklarationen (vor
allem Subtypen) sind anonym und sind in keinem <Scope> aufgelistet.

Die Einschr�nkungen, die Subtypen dem Wertebereich eines Objektes
auferlegen, werden vom Parser zwar ermittelt, bleiben aber ansonsten
v�llig unber�cksichtigt. Entsprechende �berpr�fungen (bei
Arrayindizierungen oder sogar bei jeder Rechenoperation) k�nnen im
allgemeinen Fall erst zur Laufzeit des Programms durchgef�hrt werden
und m�ssen daher vom Backend implementiert werden\footnote{Oder auch
nicht, um akzeptable Performance zu erreichen.}


\smallskip
\begin{ndesc}{\b{Type} : Declaration : PosNode : Node}{}
 Die Basis der Knotentypenhierarchie f�r VHDL Typen. Knoten f�r Subtypen
  werden auch von <Type> abgeleitet.

\begin{desc}{Type \b{get_base}();}
      Ermittelt den Basistyp dieses <Type>s. Siehe <SubType>.
     \end{desc}
    
\end{ndesc}

\smallskip
\begin{ndesc}{\b{Constraint} : PosNode : Node}{}
 Die Basis f�r die Einschr�nkungen eines <SubType>s.

\end{ndesc}

\smallskip
\shortndesc{\b{Range} : Constraint : PosNode : Node}{}

\begin{ndesc}{enum \b{RangeDirection}}{}
Die Richtung eines Bereichs.
  \shortdesc{\b{DirectionUp}}
  \shortdesc{\b{DirectionDown}}
\end{ndesc}

\smallskip
\begin{ndesc}{\b{ExplicitRange} : Range : Constraint : PosNode : Node}{}
 \shortdesc{"RangeDirection" \b{dir};}
 \shortdesc{Expr \b{first};}
 \begin{desc}{Expr \b{last};}
 Die Grenzen dieser <Range>. Die Typen der Ausdr�cke
	  passen zu dem <SubType>, der diesen <Constraint> enth�lt.
	  <first> ist die Grenze, die im VHDL-Text links neben "to"
	  oder "downto" steht, <last> steht rechts davon.
	
 \end{desc}
\end{ndesc}

\smallskip
\begin{ndesc}{\b{ArrayRange} : Range : Constraint : PosNode : Node}{}
 \shortdesc{Type \b{type};}
 \shortdesc{Object \b{array};}
 \shortdesc{"int" \b{index};}
\end{ndesc}

\smallskip
\shortndesc{\b{ArrayAttr_RANGE} : ArrayRange : Range : Constraint : PosNode : Node}{}
\smallskip
\shortndesc{\b{ArrayAttr_REVERSE_RANGE} : ArrayRange : Range : Constraint : PosNode : Node}{}
\smallskip
\begin{ndesc}{\b{PreIndexConstraint} : Constraint : PosNode : Node}{\ \internal}


 \shortdesc{PreIndexConstraint \b{next};}
\end{ndesc}

\smallskip
\begin{ndesc}{\b{PreIndexRangeConstraint} : PreIndexConstraint : Constraint : \dots}{\ \internal}


 \shortdesc{Range \b{range};}
\end{ndesc}

\smallskip
\begin{ndesc}{\b{PreIndexSubtypeConstraint} : PreIndexConstraint : Constraint : \dots}{\ \internal}


 \shortdesc{Type \b{type};}
\end{ndesc}

\smallskip
\begin{ndesc}{\b{IndexConstraint} : Constraint : PosNode : Node}{}
 Ein <IndexConstraint> legt den <SubType> f�r einen Arrayindex fest. 
 \begin{desc}{IndexConstraint \b{next};}
 Zeigt auf das n�chsten <IndexConstraint>. Die Liste ist parallel
      zu der Indexliste des Arraytyps zu dem dieser <IndexConstraint>
      geh�rt.
    
 \end{desc}
 \begin{desc}{Type \b{type};}
 Der <SubType> f�r diesen Index. Dieser <SubType> hat immer
      eine <Range>, der die Indexgrenzen festlegt.
    
 \end{desc}
\end{ndesc}

\smallskip
\begin{ndesc}{\b{IncompleteType} : Type : Declaration : PosNode : Node}{\ \internal}
 Alle unfertigen Typen m�ssen noch im selben <Scope> 
komplettiert werden, daher tauchen sie im endg�ltigen Graphen nicht auf.

\end{ndesc}

\smallskip
\begin{ndesc}{\b{SubType} : Type : Declaration : PosNode : Node}{}
 \begin{desc}{Type \b{base};}
 Der unmittelbare Basistype dieses <Subtype>. <base> kann ein
	  weiterer <SubType> sein. Der endg�ltige Basistyp kann mit
	  <get_base> (siehe <Type>) ermittelt werden.
	
 \end{desc}
 \begin{desc}{Constraint \b{constraint};}
 Die Einschr�nkungen f�r diesen <SubType>. Entweder eine
	  <Range> (f�r skalare Basistypen) oder eine
	  <IndexConstraint> Liste (f�r Arrays).
	
 \end{desc}
 \begin{desc}{Name \b{resol_func};}
 Noch nicht realisiert.
	
 \end{desc}
\end{ndesc}

\smallskip
\begin{ndesc}{\b{AccessType} : Type : Declaration : PosNode : Node}{}
 \shortdesc{Type \b{designated};}
\end{ndesc}

 Die grundlegenden Typen werden gem�� der Einteilung im <LRM> durch
  die folgenden Knotentypen klassifiziert.

\smallskip
\shortndesc{\b{ScalarType} : Type : Declaration : PosNode : Node}{}
\smallskip
\shortndesc{\b{NumericType} : ScalarType : Type : Declaration : PosNode : Node}{}
\smallskip
\shortndesc{\b{IntegerType} : NumericType : ScalarType : Type : \dots}{}


\smallskip
\shortndesc{\b{FloatingType} : NumericType : ScalarType : Type : \dots}{}


 
  \bigskip
  Die implizite Definition von <universal integer> und <universal
  real> Typen wird an passender Stelle (irgenwo im "package standard")
  explizit in den Graphen aufgenommen. 

\smallskip
\shortndesc{\b{UniversalInteger} : IntegerType : NumericType : ScalarType : Type : \dots}{}

\smallskip
\shortndesc{\b{UniversalReal} : FloatingType : NumericType : ScalarType : Type : \dots}{}

\smallskip
\begin{ndesc}{\b{PhysicalType} : NumericType : ScalarType : Type : Declaration : PosNode : Node}{}
 Ein <PhysicalType> listet einfach alle seine definierten 
  Einheiten auf.

 \begin{desc}{PhysicalUnit \b{first};}
 Die erste <PhysicalUnit> dieses Typs. 
 \end{desc}
 \begin{desc}{SubType \b{declaring_subtype};}
 Der <SubType>, der den Namen dieses Typen hat. 
 \end{desc}
\end{ndesc}

\smallskip
\begin{ndesc}{\b{PhysicalUnit} : Declaration : PosNode : Node}{}
 Jede Einheit eines <PhysicalType> wird als <Declaration> im
  <Scope> des Typs aufgenommen, um sie anhand ihres Namens finden
  zu k�nnen.

 \begin{desc}{PhysicalType \b{type};}
 Zeigt auf den zugeh�rigen <PhysicalType>.
 \end{desc}
 \begin{desc}{PhysicalLiteralRef \b{value};}
 Der Wert dieser Einheit, ausgedr�ckt durch eine andere <PhysicalUnit>
      des zugeh�rigen Typs. Die grundlegende Einheit hat $<value> == NULL$.
    
 \end{desc}
 \begin{desc}{PhysicalUnit \b{next};}
 Die n�chste Einheit des zugeh�rigen Typs. 
 \end{desc}
\end{ndesc}

\smallskip
\begin{ndesc}{\b{EnumType} : ScalarType : Type : Declaration : PosNode : Node}{}
 Ein <EnumType> listet alle seine Elemente auf.
 \shortdesc{EnumLiteral \b{first};}
\end{ndesc}

\smallskip
\begin{ndesc}{\b{EnumLiteral} : Declaration : PosNode : Node}{}
 Es k�nnen mehrer <EnumLiterals> mit der gleichen <id> in einem
  <Scope> existieren. Sie werden an ihrem <EnumType> unterschieden.

 \shortdesc{EnumType \b{type};}
 \shortdesc{EnumLiteral \b{next};}
\end{ndesc}

\smallskip
\begin{ndesc}{\b{CompositeType} : Type : Declaration : PosNode : Node}{}
 Die Basis f�r Array und Records. 
\end{ndesc}

\smallskip
\begin{ndesc}{\b{ArrayType} : CompositeType : Type : Declaration : PosNode : Node}{}
 Ein <ArrayType> beschreibt immer ein <unconstraint array>.
  Die Beschr�nkungen der Indices wird durch einen <SubType> diese
  Typs dargestellt.

 \begin{desc}{IndexType \b{first_index};}
 Die Liste der unbeschr�nkten Indextypen. Der <SubType> eines
          <ArrayType>s hat als Beschr�nkung eine parallele Liste
          von <IndexConstraints>. 
 \end{desc}
 \begin{desc}{Type \b{element_type};}
 Der Typ der Arrayelemente. 
 \end{desc}
\end{ndesc}

\smallskip
\begin{ndesc}{\b{IndexType} : PosNode : Node}{}
 Beschreibt den Typ eines einzelnen Indices. 
 \shortdesc{IndexType \b{next};}
 \shortdesc{Type \b{index_type};}
\end{ndesc}

\smallskip
\begin{ndesc}{\b{SubarrayType} : ArrayType : CompositeType : Type : Declaration : PosNode : Node}{}
 \shortdesc{ArrayType \b{complete_type};}
\end{ndesc}

\smallskip
\begin{ndesc}{\b{RecordType} : CompositeType : Type : Declaration : PosNode : Node}{}
 Ein <RecordType> enth�lt einfach eine Liste aller seiner
  Elemente.

 \shortdesc{RecordElement \b{first_element};}
\end{ndesc}

\smallskip
\begin{ndesc}{\b{RecordElement} : PosNode : Node}{}
 Ein einzelnes Element eines <RecordType>.

 \shortdesc{RecordElement \b{next};}
 \begin{desc}{"Id" \b{id};}
 Der Name dieses Elements. 
 \end{desc}
 \begin{desc}{Type \b{type};}
 Der Typ. 
 \end{desc}
\end{ndesc}

\smallskip
\begin{ndesc}{\b{FileType} : Type : Declaration : PosNode : Node}{}
 \shortdesc{Type \b{content_type};}
\end{ndesc}

\smallskip
\begin{ndesc}{\b{DummyType} : Type : Declaration : PosNode : Node}{}
 F�r noch nicht implementierte Typen. Wird irgendwann verschwinden. 
\end{ndesc}

 \subsection{Konstanten, Variablen und Signale}

\smallskip
\begin{ndesc}{\b{Object} : Declaration : PosNode : Node}{}
 Die Basis f�r Variablen, Konstanten, Signale, Files und Aliasse.

 \begin{desc}{Type \b{type};}
 Der <Type> oder <SubType> dieses Objekts. 
 \end{desc}
 \begin{desc}{Expr \b{initial_value};}
 Der Initialisierungswert, wenn vorhanden. 
 \end{desc}
\end{ndesc}

\smallskip
\begin{ndesc}{\b{Variable} : Object : Declaration : PosNode : Node}{}
 \begin{desc}{"bool" \b{shared};}
 "true", wenn dies eine <shared variable> ist. Noch nicht relisiert. 
 \end{desc}
\end{ndesc}

\smallskip
\shortndesc{\b{Constant} : Object : Declaration : PosNode : Node}{}
\smallskip
\begin{ndesc}{\b{ConstantBody} : Declaration : PosNode : Node}{}
 Liefert den Initialisierungswert f�r eine Konstante nach, falls sie
  noch nicht bei ihrer Deklaration initialisiert wurde.

 \begin{desc}{Constant \b{decl};}
 Die Konstante, die versp�tet initialisiert wird. 
 \end{desc}
 \shortdesc{Expr \b{initial_value};}
\end{ndesc}

\begin{ndesc}{enum \b{SignalKind}}{}
   Die m�glichen Varianten eines Signals.
   \shortdesc{\b{SigKind_None}}
   \shortdesc{\b{SigKind_Bus}}
   \shortdesc{\b{SigKind_Register}}
 \end{ndesc}

\smallskip
\begin{ndesc}{\b{Signal} : Object : Declaration : PosNode : Node}{}
 \shortdesc{"SignalKind" \b{signal_kind};}
\end{ndesc}

\smallskip
\begin{ndesc}{\b{GuardSignal} : Signal : Object : Declaration : PosNode : Node}{}
 Ein implizites <guard signal>. Noch nicht realisiert.

\end{ndesc}

\smallskip
\shortndesc{\b{Alias} : Object : Declaration : PosNode : Node}{}
\smallskip
\begin{ndesc}{\b{File} : Object : Declaration : PosNode : Node}{}
 Der Dateiname ist in "initial_value".

 \shortdesc{Expr \b{open_mode};}
\end{ndesc}

 \section{Funktionen und Prozeduren}

Die Deklaration einer Funktion oder Prozedur wird im Graphen immer
durch zwei Teile dargestellt: die Deklaration der Schnittstelle mit
einem <Function> oder <Procedure> Knoten und die Deklaration des
Funktions- oder Prozedurrumpfes, der die Anweisungen enth�lt, mit
einem <SubprogramBody>. Diese beiden Teile sind �ber den <continued>
Zeiger des <SubprogramBody> zusammengeh�ngt. 
\smallskip
\begin{ndesc}{\b{Subprogram} : Scope : AttributedDeclaration : Declaration : PosNode : Node}{}
 Enth�lt die Parameterliste einer Funktion oder einer
  Prozedur.

 \shortdesc{Interface \b{first_formal};}
\end{ndesc}

\smallskip
\shortndesc{\b{Procedure} : Subprogram : Scope : AttributedDeclaration : Declaration : \dots}{}


\smallskip
\begin{ndesc}{\b{Function} : Subprogram : Scope : AttributedDeclaration : Declaration : \dots}{}


 \shortdesc{"bool" \b{pure};}
 \shortdesc{Type \b{return_type};}
\end{ndesc}

\smallskip
\begin{ndesc}{\b{PredefOp} : Function : Subprogram : Scope : \dots}{}

  Die Beschreibung einer vordefinierten Funktion.
  Zu diesen Funktionen gibt es keinen <SubprogramBody>.

\end{ndesc}

\begin{ndesc}{enum \b{ObjectClass}}{}
  Die verschiedenen Arten von Objekten.
   \shortdesc{\b{ObjClass_None}}
   \shortdesc{\b{ObjClass_Signal}}
   \shortdesc{\b{ObjClass_Variable}}
   \shortdesc{\b{ObjClass_Constant}}
   \shortdesc{\b{ObjClass_File}}
 \end{ndesc}

 \begin{ndesc}{enum \b{Mode}}{}
 Die verschiedenen Modi eines Parameters.
  \shortdesc{\b{Mode_None}}
  \shortdesc{\b{Mode_In}}
  \shortdesc{\b{Mode_Out}}
  \shortdesc{\b{Mode_InOut}}
  \shortdesc{\b{Mode_Buffer}}
  \shortdesc{\b{Mode_Linkage}}
 \end{ndesc}

\smallskip
\begin{ndesc}{\b{Interface} : Object : Declaration : PosNode : Node}{}
 Ein Parameter eines <Subprogram>s.

 \begin{desc}{Interface \b{next_element};}
 Der n�chste Parameter in der Liste. 
 \end{desc}
 \shortdesc{"ObjectClass" \b{object_class};}
 \shortdesc{"Mode" \b{mode};}
 \shortdesc{"bool" \b{buffer};}
\end{ndesc}

\smallskip
\begin{ndesc}{\b{SubprogramBody} : Scope : AttributedDeclaration : Declaration : \dots}{}

  Die Anweisungen zu einem <Subprogram>. Der <continued>-Zeiger
  dieses <Scopes> zeigt auf die zugeh�rige <Subprogram>-Deklaration.

 \shortdesc{Statement \b{stats};}
\end{ndesc}

 \section{Ausdr�cke}

\smallskip
\begin{ndesc}{\b{Expr} : PosNode : Node}{}
 Die Basis f�r alle Ausdrucksknoten.

\end{ndesc}

\smallskip
\begin{ndesc}{\b{UnresolvedName} : Expr : PosNode : Node}{\ \internal}

 \shortdesc{Name \b{name};}
\end{ndesc}

\smallskip
\begin{ndesc}{\b{FunctionCall} : Expr : PosNode : Node}{}
 Ein Funktionsaufruf.

 \begin{desc}{Function \b{func};}
 Die aufgerufene Funktion. 
 \end{desc}
 \begin{desc}{Association \b{first_actual};}
 Ein Ausdruck f�r jeden Parameter der Funktion. Diese
          Ausdr�cke werden nicht in der Reihenfolge der Parameter der Funktion
          aufgelistet, sondern in der Reihenfolge, in der sie im VHDL-Text
          stehen. 
 \end{desc}
\end{ndesc}

\smallskip
\begin{ndesc}{\b{ProcedureCall} : Expr : PosNode : Node}{\ \internal}

 \shortdesc{Procedure \b{proc};}
 \shortdesc{Association \b{first_actual};}
\end{ndesc}

\smallskip
\begin{ndesc}{\b{AmbgCall} : Expr : PosNode : Node}{\ \internal}

 \shortdesc{NamedAssocElem \b{first_actual};}
\end{ndesc}

\smallskip
\begin{ndesc}{\b{GenAssocElem} : PosNode : Node}{\ \internal}

 \shortdesc{GenAssocElem \b{next};}
\end{ndesc}

\smallskip
\begin{ndesc}{\b{NamedAssocElem} : GenAssocElem : PosNode : Node}{\ \internal}

 \shortdesc{Name \b{formal};}
 \shortdesc{Expr \b{actual};}
\end{ndesc}

\smallskip
\begin{ndesc}{\b{SubtypeAssocElem} : GenAssocElem : PosNode : Node}{\ \internal}

 \shortdesc{Type \b{type};}
\end{ndesc}

\smallskip
\begin{ndesc}{\b{RangeAssocElem} : GenAssocElem : PosNode : Node}{\ \internal}

 \shortdesc{Range \b{range};}
\end{ndesc}

\smallskip
\begin{ndesc}{\b{Association} : Node}{}
 Eine Verbindung zwischen einem Schnittstellenobjekt und einem Ausdruck.

 \shortdesc{Association \b{next};}
 \begin{desc}{Interface \b{formal};}
 Die Deklaration des Schnittstellenobjekts. 
 \end{desc}
 \begin{desc}{Declaration \b{formal_conversion};}
 Noch nicht realisiert. 
 \end{desc}
 \begin{desc}{Expr \b{actual};}
 Der Ausdruck, der <formal> zugeordnet wird. 
 \end{desc}
 \begin{desc}{Declaration \b{actual_conversion};}
 noch nicht realisiert. 
 \end{desc}
\end{ndesc}

\smallskip
\begin{ndesc}{\b{TypeConversion} : Expr : PosNode : Node}{}
 Eine explizite Typumwandlung. 
 \begin{desc}{Type \b{target_type};}
 Der Typ, in dem umgewandelt werden soll.
 \end{desc}
 \begin{desc}{Expr \b{expression};}
 Der Ausdruck, der umgewandelt werden soll.
	  Der Typ dieses Ausdrucks und <target_type> sind
	  <closely related>. Eine Umwandlung ist also m�glich.
	  (Wird noch nicht �berpr�ft)
	
 \end{desc}
\end{ndesc}

\smallskip
\begin{ndesc}{\b{QualifiedExpr} : Expr : PosNode : Node}{}
 Ein freundlicher und zutreffender Hinweis auf den Typ eines Ausdrucks. 
 \begin{desc}{Type \b{type};}
 Der Typ von <expression>. 
 \end{desc}
 \begin{desc}{Expr \b{expression};}
 Dieser Ausdruck hat den Typ <type>. 
 \end{desc}
\end{ndesc}

\smallskip
\begin{ndesc}{\b{NewExpr} : Expr : PosNode : Node}{}
 \shortdesc{Type \b{type};}
 \shortdesc{Expr \b{initial_value};}
\end{ndesc}

\smallskip
\begin{ndesc}{\b{PrimaryExpr} : Expr : PosNode : Node}{}
 Die Basis f�r alle Ausdr�cke, die kein Funktionsaufruf sind
 (Bl�tter des Ausdruckbaums).

\end{ndesc}

\smallskip
\shortndesc{\b{OpenExpr} : PrimaryExpr : Expr : PosNode : Node}{}
\begin{ndesc}{typedef char *\b{Literal}}{}
   Der Typ eines <abstract literals>. Der Parser selbst wandelt diese
   Konstanten nicht in numerische Typen (z.B. "long" oder "double"), um das
   Backend nicht auf eine Representation festzunageln.
 \end{ndesc}

\smallskip
\begin{ndesc}{\b{LiteralRef} : PrimaryExpr : Expr : PosNode : Node}{}
 Eine 'literarische' Konstante. 
 \begin{desc}{"Literal" \b{value};}
 Der Wert der Konstante als String. 
 \end{desc}
\end{ndesc}

\smallskip
\begin{ndesc}{\b{AmbgArrayLitRef} : PrimaryExpr : Expr : PosNode : Node}{\ \internal}

 \shortdesc{"Literal" \b{value};}
\end{ndesc}

\smallskip
\begin{ndesc}{\b{ArrayLiteralRef} : PrimaryExpr : Expr : PosNode : Node}{}
 Eine 'literarische' Arraykonstante (<bit string literal>). 
 \begin{desc}{"Literal" \b{value};}
 Der Wert der Konstanten als String. 
 \end{desc}
 \begin{desc}{Type \b{type};}
 Zus�tzlich zum Wert wird noch der Typ der Arraykonstanten angegeben.
	  Dieser Typ beinhaltet in einem <SubType> auch die Ausdehnung dieses
	  Arrays. 
 \end{desc}
\end{ndesc}

\smallskip
\begin{ndesc}{\b{PhysicalLiteralRef} : LiteralRef : PrimaryExpr : Expr : PosNode : Node}{}
 Eine physikalische Konstante. 
 \begin{desc}{PhysicalUnit \b{unit};}
 Die Einheit dieser Konstanten. Der Zahlenwert steht
	  <value> der <LiteralRef>-Basis. 
 \end{desc}
\end{ndesc}

\smallskip
\shortndesc{\b{AmbgNullExpr} : \dots}{}

\smallskip
\begin{ndesc}{\b{NullExpr} : PrimaryExpr : Expr : PosNode : Node}{}
 \shortdesc{Type \b{type};}
\end{ndesc}

\smallskip
\begin{ndesc}{\b{ObjectRef} : PrimaryExpr : Expr : PosNode : Node}{}
 Die Basis f�r Zugriffe auf <Object>s.
  Ein <ObjectRef> stellt einige Funktionen zur Verf�gung, die von
  den konkreten Zugriffen (auf Variablen, Arrayelemente, \dots)
  passend implementiert werden.

\begin{desc}{virtual ObjectClass \b{get_class}();}
	  Ermittelt die Klasse des Objekts, auf das zugegriffen wird.
	 \end{desc}
	 \begin{desc}{virtual ObjectClass \b{get_mode}();}
	  Ermittelt den Modus.
	 \end{desc}
	 \begin{desc}{virtual ObjectClass \b{get_type}();}
	  Ermittelt den Typ.
	 \end{desc}
	 \begin{desc}{virtual ObjectClass \b{is_constant}();}
	   �quivalent zu "get_class() == ObjClass_Constant".
	 \end{desc}
	 \begin{desc}{virtual ObjectClass \b{is_variable}();}
	   �quivalent zu "get_class() == ObjClass_Variable".
	 \end{desc}
	 \begin{desc}{virtual ObjectClass \b{is_signal}();}
	   �quivalent zu "get_class() == ObjClass_Signal".
	 \end{desc}
	 \begin{desc}{virtual ObjectClass \b{is_file}();}
	   �quivalent zu "get_class() == ObjClass_File".
	 \end{desc}
	 \begin{desc}{virtual ObjectClass \b{is_readable}();}
	  Ermittelt, ob das Objekt gelesen werden kann.
	 \end{desc}
	 \begin{desc}{virtual ObjectClass \b{is_writeable}();}
	  Ermittelt, ob das Objekt geschrieben werden kann.
	 \end{desc}
	
\end{ndesc}

\smallskip
\begin{ndesc}{\b{SimpleObjectRef} : ObjectRef : PrimaryExpr : Expr : PosNode : Node}{}
 Ein Zugriff auf ein 'normales' Objekt (kein Array- oder Recordelement).

 \begin{desc}{Object \b{object};}
 Das Objekt h�chstselbst. 
 \end{desc}
 \begin{desc}{Name \b{name};}
 Der Name, der im VHDL-Text stand. 
 \end{desc}
\end{ndesc}

\smallskip
\begin{ndesc}{\b{AccessObjectRef} : ObjectRef : PrimaryExpr : Expr : PosNode : Node}{}
 Ein Zugriff auf ein Objekt durch ein <access value>.

 \begin{desc}{Expr \b{access};}
 Ein Ausdruck f�r den <acces value>. 
 \end{desc}
 \shortdesc{Type \b{accessed_type};}
\end{ndesc}

\smallskip
\begin{ndesc}{\b{RecordObjectRef} : ObjectRef : PrimaryExpr : Expr : PosNode : Node}{}
 Ein Zugriff auf ein Recordelement.

 \begin{desc}{Expr \b{record};}
 Das Recordobjekt als ganzes. 
 \end{desc}
 \begin{desc}{RecordType \b{record_type};}
 Der Typ von <record>. 
 \end{desc}
 \begin{desc}{RecordElement \b{element};}
 Das Element des Records, auf das zugegriffen wird. 
 \end{desc}
\end{ndesc}

\smallskip
\begin{ndesc}{\b{GenericArrayObjectRef} : ObjectRef : PrimaryExpr : Expr : \dots}{}
 Die Basis f�r einen Zugriff auf einen Teil eines Arrays.

 \begin{desc}{Expr \b{array};}
 Das Array als ganzes. 
 \end{desc}
 \begin{desc}{ArrayType \b{array_type};}
 Der Typ dieses Arrays. Er k�nnte theoretisch auch
          aus <array> bestimmt werden. Das ist aber nicht ganz trivial.
	
 \end{desc}
\end{ndesc}

\smallskip
\begin{ndesc}{\b{ArrayObjectRef} : GenericArrayObjectRef : ObjectRef : \dots}{}
 Ein Zugriff auf ein Arrayelement.

 \begin{desc}{IndexValue \b{first_index};}
 Ein Wert f�r jeden Index des Arrays. Die Liste ist parallel zur
	  Indexliste von <array_type>.
	
 \end{desc}
\end{ndesc}

\smallskip
\begin{ndesc}{\b{IndexValue} : PosNode : Node}{}
 \shortdesc{IndexValue \b{next};}
 \shortdesc{Expr \b{index};}
\end{ndesc}

\smallskip
\begin{ndesc}{\b{SliceObjectRef} : GenericArrayObjectRef : ObjectRef : \dots}{}
 Ein Zugriff auf eine Scheibe eines Arrays.

 \begin{desc}{Type \b{slice};}
 Der <Range> dieses Subtypes gibt den Bereich der
	  Scheibe an.
	
 \end{desc}
\end{ndesc}

\smallskip
\begin{ndesc}{\b{EnumLitRef} : PrimaryExpr : Expr : PosNode : Node}{}
 Ein Zugriff aif eine Enumerationskonstante. 
 \begin{desc}{EnumLiteral \b{value};}
 Die Konstante selbst. 
 \end{desc}
\end{ndesc}

\smallskip
\shortndesc{\b{AmbgEnumLitRef} : PrimaryExpr : Expr : PosNode : Node}{\ \internal}

\smallskip
\begin{ndesc}{\b{AttrSignalRef} : ObjectRef : PrimaryExpr : Expr : PosNode : Node}{}
 Ein Zugriff auf ein eingebautes Attribut eines Signals, das
  ein Signal als Wert hat.
  Konkrete Attribute werden durch abgeleitete Knoten unterschieden.

 \begin{desc}{Type \b{attr_type};}
 Der Type des Attributes. 
 \end{desc}
 \begin{desc}{ObjectRef \b{signal};}
 Das Signal, auf das sich dieses Attribut bezieht. 
 \end{desc}
\end{ndesc}

\smallskip
\begin{ndesc}{\b{Attr_DELAYED} : AttrSignalRef : ObjectRef : \dots}{}


 \begin{desc}{Expr \b{delay};}
 Der Parameter des "DELAY" Attributes, oder "NULL" wenn keiner
	  angegeben wurde. 
 \end{desc}
\end{ndesc}

\smallskip
\begin{ndesc}{\b{Attr_STABLE} : AttrSignalRef : ObjectRef : \dots}{}


 \begin{desc}{Expr \b{duration};}
 Der Parameter des "STABLE" Attributes, oder "NULL" wenn keiner
	  angegeben wurde. 
 \end{desc}
\end{ndesc}

\smallskip
\begin{ndesc}{\b{Attr_QUIET} : AttrSignalRef : ObjectRef : \dots}{}


 \begin{desc}{Expr \b{duration};}
 Der Parameter des "QUIT" Attributes, oder "NULL" wenn keiner
	  angegeben wurde. 
 \end{desc}
\end{ndesc}

\smallskip
\shortndesc{\b{Attr_TRANSACTION} : AttrSignalRef : ObjectRef : \dots}{}


\smallskip
\begin{ndesc}{\b{AttrFunctionCall} : Expr : PosNode : Node}{}
 Ein Aufruf eines Attributes, das eine Funktion darstellt.

 \begin{desc}{Type \b{attr_type};}
 Der Typ dieses Attributes (der R�ckgabetyp der Funktion). 
 \end{desc}
\end{ndesc}

\smallskip
\begin{ndesc}{\b{AttrSigFunc} : AttrFunctionCall : Expr : PosNode : Node}{}
 Ein Aufruf eines Attributes, das eine Funktion darstellt und sich auf ein
  Signal bezieht.

 \begin{desc}{ObjectRef \b{signal};}
 Ein Zugriff auf das Signal. 
 \end{desc}
\end{ndesc}

\smallskip
\shortndesc{\b{Attr_EVENT} : AttrSigFunc : AttrFunctionCall : \dots}{}


\smallskip
\shortndesc{\b{Attr_ACTIVE} : AttrSigFunc : AttrFunctionCall : \dots}{}


\smallskip
\shortndesc{\b{Attr_LAST_EVENT} : AttrSigFunc : AttrFunctionCall : \dots}{}


\smallskip
\shortndesc{\b{Attr_LAST_ACTIVE} : AttrSigFunc : AttrFunctionCall : \dots}{}


\smallskip
\shortndesc{\b{Attr_LAST_VALUE} : AttrSigFunc : AttrFunctionCall : \dots}{}


\smallskip
\shortndesc{\b{Attr_DRIVING} : AttrSigFunc : AttrFunctionCall : \dots}{}


\smallskip
\shortndesc{\b{Attr_DRIVING_VALUE} : AttrSigFunc : AttrFunctionCall : \dots}{}


\smallskip
\begin{ndesc}{\b{AttrTypeFunc} : AttrFunctionCall : Expr : PosNode : Node}{}
 Ein Aufruf eines Aggregates, das eine Funktion darstellt und sich auf einen
  Typen bezieht.

 \begin{desc}{Type \b{prefix};}
 Der Prefix des Attributes. 
 \end{desc}
 \begin{desc}{Expr \b{argument};}
 Das Argument. 
 \end{desc}
\end{ndesc}

\smallskip
\shortndesc{\b{Attr_LEFT} : AttrTypeFunc : AttrFunctionCall : \dots}{}


\smallskip
\shortndesc{\b{Attr_RIGHT} : AttrTypeFunc : AttrFunctionCall : \dots}{}


\smallskip
\shortndesc{\b{Attr_LOW} : AttrTypeFunc : AttrFunctionCall : \dots}{}


\smallskip
\shortndesc{\b{Attr_HIGH} : AttrTypeFunc : AttrFunctionCall : \dots}{}


\smallskip
\shortndesc{\b{Attr_ASCENDING} : AttrTypeFunc : AttrFunctionCall : \dots}{}


\smallskip
\shortndesc{\b{Attr_IMAGE} : AttrTypeFunc : AttrFunctionCall : \dots}{}


\smallskip
\shortndesc{\b{Attr_VALUE} : AttrTypeFunc : AttrFunctionCall : \dots}{}


\smallskip
\shortndesc{\b{Attr_POS} : AttrTypeFunc : AttrFunctionCall : \dots}{}


\smallskip
\shortndesc{\b{Attr_VAL} : AttrTypeFunc : AttrFunctionCall : \dots}{}


\smallskip
\shortndesc{\b{Attr_SUCC} : AttrTypeFunc : AttrFunctionCall : \dots}{}


\smallskip
\shortndesc{\b{Attr_PRED} : AttrTypeFunc : AttrFunctionCall : \dots}{}


\smallskip
\shortndesc{\b{Attr_LEFTOF} : AttrTypeFunc : AttrFunctionCall : \dots}{}


\smallskip
\shortndesc{\b{Attr_RIGHTOF} : AttrTypeFunc : AttrFunctionCall : \dots}{}


\smallskip
\begin{ndesc}{\b{ArrayAttr} : Expr : PosNode : Node}{}
 \shortdesc{Type \b{attr_type};}
 \shortdesc{Object \b{array};}
 \shortdesc{"int" \b{index};}
\end{ndesc}

\smallskip
\shortndesc{\b{ArrayAttr_LEFT} : ArrayAttr : Expr : PosNode : Node}{}
\smallskip
\shortndesc{\b{ArrayAttr_RIGHT} : ArrayAttr : Expr : PosNode : Node}{}
\smallskip
\shortndesc{\b{ArrayAttr_HIGH} : ArrayAttr : Expr : PosNode : Node}{}
\smallskip
\shortndesc{\b{ArrayAttr_LOW} : ArrayAttr : Expr : PosNode : Node}{}
\smallskip
\shortndesc{\b{ArrayAttr_ASCENDING} : ArrayAttr : Expr : PosNode : Node}{}
\smallskip
\shortndesc{\b{ArrayAttr_LENGTH} : ArrayAttr : Expr : PosNode : Node}{}
\smallskip
\begin{ndesc}{\b{AttributeSpec} : PosNode : Node}{}
 \shortdesc{"Id" \b{attr_desig};}
 \shortdesc{EntitySpec \b{entities};}
 \shortdesc{Expr \b{value};}
\end{ndesc}

\smallskip
\begin{ndesc}{\b{EntitySpec} : PosNode : Node}{}
 \shortdesc{EntityNameList \b{names};}
 \shortdesc{"int" \b{entity_class};}
\end{ndesc}

\smallskip
\shortndesc{\b{EntityNameList} : PosNode : Node}{}
\smallskip
\begin{ndesc}{\b{EntityNameList_Ids} : EntityNameList : PosNode : Node}{}
 \shortdesc{IdList \b{ids};}
\end{ndesc}

\smallskip
\shortndesc{\b{EntityNameList_ALL} : EntityNameList : PosNode : Node}{}
\smallskip
\shortndesc{\b{EntityNameList_OTHERS} : EntityNameList : PosNode : Node}{}
\smallskip
\begin{ndesc}{\b{Aggregate} : Expr : PosNode : Node}{}
 \shortdesc{Type \b{type};}
\end{ndesc}

\smallskip
\begin{ndesc}{\b{RecordAggregate} : Aggregate : Expr : PosNode : Node}{}
 \shortdesc{RecAggrAssoc \b{first_assoc};}
\end{ndesc}

\smallskip
\begin{ndesc}{\b{RecAggrAssoc} : PosNode : Node}{}
 \shortdesc{RecAggrAssoc \b{next};}
 \shortdesc{RecordElement \b{elem};}
 \shortdesc{Expr \b{actual};}
\end{ndesc}

\smallskip
\begin{ndesc}{\b{ArrayAggregate} : Aggregate : Expr : PosNode : Node}{}
 \shortdesc{ArrAggrAssoc \b{first_assoc};}
\end{ndesc}

\smallskip
\begin{ndesc}{\b{ArrAggrAssoc} : PosNode : Node}{}
 \shortdesc{ArrAggrAssoc \b{next};}
 \shortdesc{Expr \b{actual};}
\end{ndesc}

\smallskip
\begin{ndesc}{\b{SingleArrAggrAssoc} : ArrAggrAssoc : PosNode : Node}{}
 \shortdesc{Expr \b{index};}
\end{ndesc}

\smallskip
\begin{ndesc}{\b{RangeArrAggrAssoc} : ArrAggrAssoc : PosNode : Node}{}
 \shortdesc{Range \b{range};}
\end{ndesc}

\smallskip
\shortndesc{\b{SliceArrAggrAssoc} : RangeArrAggrAssoc : ArrAggrAssoc : PosNode : Node}{}
\smallskip
\shortndesc{\b{OthersArrAggrAssoc} : ArrAggrAssoc : PosNode : Node}{}
\smallskip
\begin{ndesc}{\b{AmbgAggregate} : Expr : PosNode : Node}{}
 \shortdesc{ElemAssoc \b{first_assoc};}
\end{ndesc}

\smallskip
\shortndesc{\b{ArtificialAmbgAggregate} : AmbgAggregate : Expr : PosNode : Node}{}
\smallskip
\begin{ndesc}{\b{ElemAssoc} : PosNode : Node}{}
 \shortdesc{ElemAssoc \b{next};}
 \shortdesc{Choice \b{first_choice};}
 \shortdesc{Expr \b{actual};}
\end{ndesc}

\smallskip
\begin{ndesc}{\b{Choice} : PosNode : Node}{}
 \shortdesc{Choice \b{next};}
\end{ndesc}

\smallskip
\begin{ndesc}{\b{ExprChoice} : Choice : PosNode : Node}{}
 \shortdesc{Expr \b{expr};}
\end{ndesc}

\smallskip
\begin{ndesc}{\b{RangeChoice} : Choice : PosNode : Node}{}
 \shortdesc{GenAssocElem \b{range};}
 \shortdesc{"bool" \b{actual_is_slice};}
\end{ndesc}

\smallskip
\begin{ndesc}{\b{NamedChoice} : Choice : PosNode : Node}{}
 \shortdesc{Name \b{name};}
\end{ndesc}

\smallskip
\shortndesc{\b{OthersChoice} : Choice : PosNode : Node}{}
 \section{Sequentielle Anweisungen}

Die sequentiellen Anweisungen, die hintereinander ausgef�hrt werden
sollen, werden zu einer Liste zusammengeh�ngt. Zeiger auf eine
Anweisung, bei denen in der Beschreibung von mehreren Anweisungen die
Rede ist, meinen die komplette Liste.


\smallskip
\begin{ndesc}{\b{Label} : Declaration : PosNode : Node}{}
Die Namen von benannten Anweisungen (<labels>) werden als <Declaration>s
 in den zu der Anweisung geh�renden <Scope> aufgenommen.

 \begin{desc}{Statement \b{stat};}
 Die Anweisung, die diesen Namen tr�gt. 
 \end{desc}
\end{ndesc}

\smallskip
\begin{ndesc}{\b{Statement} : PosNode : Node}{}
 Die Basis f�r die verschiedenen Anweisungen. 
 \begin{desc}{Label \b{label};}
 Falls diese Anweisung einen Namen hat: hier ist er. Falls nicht,
	  ist "<label> == NULL". 
 \end{desc}
 \begin{desc}{Statement \b{next};}
 Die n�chste Anweisung. 
 \end{desc}
\end{ndesc}

\smallskip
\begin{ndesc}{\b{DummyStat} : Statement : PosNode : Node}{}
 F�r unimplementierte Anweisungen. Verschwindet irgendwann. 

\end{ndesc}

\smallskip
\begin{ndesc}{\b{ReturnStat} : Statement : PosNode : Node}{}
 Eine "return"-Anweisung.

 \begin{desc}{Expr \b{value};}
 Der Wert, der zur�ckgegeben werden soll oder "NULL", falls
	  diese Anweisung zu einer Prozedur geh�rt und nicht zu einer
	  Funktion. 
 \end{desc}
\end{ndesc}

\smallskip
\begin{ndesc}{\b{VarAssignment} : Statement : PosNode : Node}{}
 Eine Variablenzuweisung.

 \begin{desc}{Expr \b{target};}
 Das Ding, an das zugewiesen werden soll. 
 \end{desc}
 \begin{desc}{Expr \b{value};}
 Der neue Wert. 
 \end{desc}
\end{ndesc}

\smallskip
\begin{ndesc}{\b{IfStat} : Statement : PosNode : Node}{}
 Eine "if"-Anweisung.

 \begin{desc}{Expr \b{cond};}
 Die Bedingung. Der Ausdruck ist immer vom Typ "bool". 
 \end{desc}
 \begin{desc}{Statement \b{then_stats};}
 Die Anweisungen, die ausgef�hrt werden sollen, wenn
	  "<cond> == true" ist. 
 \end{desc}
 \begin{desc}{Statement \b{else_stats};}
 Die Anweisungen, die ausgef�hrt werden sollen, wenn
	  "<cond> != true" ist. 
 \end{desc}
\end{ndesc}

\smallskip
\begin{ndesc}{\b{CaseStat} : Statement : PosNode : Node}{}
 \shortdesc{Expr \b{switch_expr};}
 \shortdesc{CaseAlternative \b{first_alternative};}
\end{ndesc}

\smallskip
\begin{ndesc}{\b{CaseAlternative} : PosNode : Node}{}
 \shortdesc{CaseAlternative \b{next};}
 \shortdesc{Choice \b{first_choice};}
 \shortdesc{Statement \b{stats};}
\end{ndesc}

\smallskip
\begin{ndesc}{\b{LoopStat} : Statement : PosNode : Node}{}
 Eine Schleife ist ein G�ltigkeitsbereich und gleichzeitig eine
  Anweisung. Mangels Mehrfachvererbung wird der <Scope> einer Schleife
  durch einen separaten <LoopScope> Knoten realisiert.

 \begin{desc}{LoopScope \b{scope};}
 Der zugeh�rige <Scope>. 
 \end{desc}
 \begin{desc}{IterationScheme \b{iteration_scheme};}
 Das Schleifenschema. 
 \end{desc}
 \begin{desc}{Statement \b{stats};}
 Die abh�ngigen Anweisungen. 
 \end{desc}
\end{ndesc}

\smallskip
\begin{ndesc}{\b{LoopScope} : Scope : AttributedDeclaration : Declaration : PosNode : Node}{}
 Der zu einer Schleife geh�rende <Scope>.

 \begin{desc}{LoopStat \b{loop};}
 Die zugeh�rige Schleifenanweisung. 
 \end{desc}
\end{ndesc}

\smallskip
\begin{ndesc}{\b{IterationScheme} : PosNode : Node}{}
 Die Basis f�r alle Schleifenschema. 

\end{ndesc}

\smallskip
\begin{ndesc}{\b{WhileScheme} : IterationScheme : PosNode : Node}{}
 Ein "while"-Schema.

 \begin{desc}{Expr \b{condition};}
 Die Bedingung. Solange "<condition> == true", sollen
	  die abh�ngigen Anweisungen ausgef�hrt werden. 
 \end{desc}
\end{ndesc}

\smallskip
\begin{ndesc}{\b{ForScheme} : IterationScheme : PosNode : Node}{}
 Ein "for"-Schema. 
 \begin{desc}{Object \b{var};}
 Die Laufvariable. Die Deklaration von <var> steht im
	  <LoopScope> dieser Schleife. 
 \end{desc}
 \begin{desc}{Range \b{range};}
 Der Bereich, �ber den <var> laufen soll. 
 \end{desc}
\end{ndesc}

\smallskip
\begin{ndesc}{\b{PreForScheme} : IterationScheme : PosNode : Node}{\ \internal}


 \shortdesc{"Id" \b{var};}
 \shortdesc{PreIndexConstraint \b{range};}
\end{ndesc}

\smallskip
\begin{ndesc}{\b{LoopControlStat} : Statement : PosNode : Node}{}
 Die Basis f�r die "exit"- und "next"-Anweisungen.

 \begin{desc}{LoopStat \b{loop};}
 Die Schleife, auf die sich die Anweisung bezieht. 
 \end{desc}
 \begin{desc}{Expr \b{when};}
 Die Bedingung f�r die tats�chliche Ausf�hrung der Anweisung. 
 \end{desc}
\end{ndesc}

\smallskip
\begin{ndesc}{\b{NextStat} : LoopControlStat : Statement : PosNode : Node}{}
 Eine "next"-Anweisung. 

\end{ndesc}

\smallskip
\begin{ndesc}{\b{ExitStat} : LoopControlStat : Statement : PosNode : Node}{}
 Eine "exit"-Anweisung.

\end{ndesc}

\smallskip
\begin{ndesc}{\b{NullStat} : Statement : PosNode : Node}{}
 Eine "null"-Anweisung.

\end{ndesc}

\smallskip
\begin{ndesc}{\b{ProcedureCallStat} : Statement : PosNode : Node}{}
 Ein Prozeduraufruf. Analog zu einem <FunctionCall>, siehe auch dort.

 \begin{desc}{Procedure \b{proc};}
 Die Prozedur, die aufgerufen werden soll. 
 \end{desc}
 \begin{desc}{Association \b{first_actual};}
 Ein Ausdruck f�r jeden Parameter der Prozedur. Diese
          Ausdr�cke werden nicht in der Reihenfolge der Parameter der Funktion
          aufgelistet, sondern in der Reihenfolge, in der sie im VHDL-Text
          stehen. 
 \end{desc}
\end{ndesc}

\smallskip
\begin{ndesc}{\b{WaitStat} : Statement : PosNode : Node}{}
 Ein "wait"-Statement.

 \begin{desc}{SignalList \b{first_sensitivity};}
 Die Signale, auf die reagiert werden soll. 
 \end{desc}
 \shortdesc{Expr \b{condition};}
 \shortdesc{Expr \b{timeout};}
\end{ndesc}

\smallskip
\begin{ndesc}{\b{SignalList} : Node}{}
 \shortdesc{ObjectRef \b{signal};}
 \shortdesc{SignalList \b{next};}
\end{ndesc}

\smallskip
\begin{ndesc}{\b{WaveformElement} : Node}{}
 Ein Teil einer Wellenform. 

 \begin{desc}{Expr \b{value};}
 Der Wert dieses Teils. 
 \end{desc}
 \begin{desc}{Expr \b{after};}
 Die Zeit, bis es soweit ist. 
 \end{desc}
 \begin{desc}{WaveformElement \b{next};}
 Das n�chste Element der kompletten Wellenform. 
 \end{desc}
\end{ndesc}

\smallskip
\begin{ndesc}{\b{SignalAssignment} : Statement : PosNode : Node}{}
 Eine Signalzuweisung.

 \begin{desc}{Expr \b{target};}
 Das Signal, an das zugewiesen wird. 
 \end{desc}
 \shortdesc{"bool" \b{transport};}
 \begin{desc}{WaveformElement \b{first_wave};}
 Die Wellenform. 
 \end{desc}
\end{ndesc}

\smallskip
\begin{ndesc}{\b{AssertStat} : Statement : PosNode : Node}{}
 \shortdesc{Expr \b{condition};}
 \shortdesc{Expr \b{report};}
 \shortdesc{Expr \b{severity};}
\end{ndesc}

 \section {Parallele Anweisungen} 

Parellele Anweisungen sind -- wie sequentielle Anweisungen -- zu einer
Liste zusammengeh�ngt.

\smallskip
\begin{ndesc}{\b{ConcurrentStatement} : Scope : AttributedDeclaration : \dots}{}
 Die Basis f�r die parallelen Anweisungen. <ConcurrentStatement> sind
  von <Scope> abgeleitet, da einige von ihnen Deklarationen enthalten.

 \begin{desc}{ConcurrentStatement \b{next_stat};}
 Das n�chste Statement in der Liste. 
 \end{desc}
\end{ndesc}

\smallskip
\begin{ndesc}{\b{Process} : ConcurrentStatement : Scope : AttributedDeclaration : \dots}{}


 \shortdesc{SignalList \b{sensitivities};}
 \begin{desc}{Statement \b{stats};}
 Die sequentiellen Anweisungen innerhalb des Prozesses. 
 \end{desc}
\end{ndesc}

\smallskip
\begin{ndesc}{\b{CondalWaveform} : PosNode : Node}{\ \internal}


 \shortdesc{WaveformElement \b{wave};}
 \shortdesc{Expr \b{condition};}
 \shortdesc{CondalWaveform \b{else_wave};}
\end{ndesc}

\smallskip
\begin{ndesc}{\b{CondalSignalAssign} : PosNode : Node}{\ \internal}


 \shortdesc{Expr \b{target};}
 \shortdesc{"bool" \b{transport};}
 \shortdesc{CondalWaveform \b{wave};}
\end{ndesc}

 \section{Komponenten}


\smallskip
\begin{ndesc}{\b{Component} : ConcurrentStatement : Scope : AttributedDeclaration : Declaration : PosNode : Node}{}
 \shortdesc{Interface \b{first_generic};}
 \shortdesc{Interface \b{first_port};}
\end{ndesc}

\smallskip
\begin{ndesc}{\b{Block} : Component : ConcurrentStatement : Scope : AttributedDeclaration : Declaration : PosNode : Node}{}
 \shortdesc{ConcurrentStatement \b{stats};}
 \shortdesc{ConfigSpec \b{specs};}
\end{ndesc}

\smallskip
\shortndesc{\b{Entity} : Block : Component : ConcurrentStatement : Scope : AttributedDeclaration : Declaration : PosNode : Node}{}
\smallskip
\shortndesc{\b{Architecture} : Block : Component : ConcurrentStatement : Scope : AttributedDeclaration : Declaration : PosNode : Node}{}
\smallskip
\begin{ndesc}{\b{BlockStat} : Block : Component : ConcurrentStatement : Scope : AttributedDeclaration : Declaration : PosNode : Node}{}
 \shortdesc{BindingIndic \b{binding};}
\end{ndesc}

\smallskip
\begin{ndesc}{\b{ComponentInst} : ConcurrentStatement : Scope : AttributedDeclaration : Declaration : PosNode : Node}{}
 \shortdesc{BindingIndic \b{binding};}
 \shortdesc{ConfigSpec \b{config};}
\end{ndesc}

\smallskip
\begin{ndesc}{\b{ConfigSpec} : PosNode : Node}{}
 \shortdesc{ComponentSpec \b{comps};}
 \shortdesc{BindingIndic \b{binding};}
 \shortdesc{ConfigSpec \b{next};}
\end{ndesc}

\smallskip
\begin{ndesc}{\b{ComponentSpec} : Node}{}
 \shortdesc{InstList \b{ids};}
 \shortdesc{Component \b{comp};}
\end{ndesc}

\smallskip
\shortndesc{\b{InstList} : PosNode : Node}{}
\smallskip
\begin{ndesc}{\b{InstList_Ids} : InstList : PosNode : Node}{}
 \shortdesc{IdList \b{ids};}
\end{ndesc}

\smallskip
\shortndesc{\b{InstList_ALL} : InstList : PosNode : Node}{}
\smallskip
\shortndesc{\b{InstList_OTHERS} : InstList : PosNode : Node}{}
\smallskip
\begin{ndesc}{\b{BindingIndic} : Node}{}
 \shortdesc{Component \b{unit};}
 \shortdesc{Association \b{generic_assoc};}
 \shortdesc{Association \b{port_assoc};}
\end{ndesc}

\smallskip
\begin{ndesc}{\b{IncrementalBindingIndic} : Node}{}
 \shortdesc{Component \b{unit};}
 \shortdesc{NamedAssocElem \b{generic_assoc};}
 \shortdesc{NamedAssocElem \b{port_assoc};}
\end{ndesc}

\smallskip
\begin{ndesc}{\b{Configuration} : Component : ConcurrentStatement : Scope : AttributedDeclaration : Declaration : PosNode : Node}{}
 \shortdesc{Entity \b{entity};}
 \shortdesc{BlockConfig \b{config};}
\end{ndesc}

\smallskip
\begin{ndesc}{\b{BaseConfig} : Scope : AttributedDeclaration : Declaration : PosNode : Node}{}
 \shortdesc{BaseConfig \b{next_config};}
\end{ndesc}

\smallskip
\begin{ndesc}{\b{BlockConfig} : BaseConfig : Scope : AttributedDeclaration : Declaration : PosNode : Node}{}
 \shortdesc{Block \b{block};}
 \shortdesc{BaseConfig \b{configs};}
\end{ndesc}

\smallskip
\begin{ndesc}{\b{CompConfig} : BaseConfig : Scope : AttributedDeclaration : Declaration : PosNode : Node}{}
 \shortdesc{ComponentSpec \b{comp_spec};}
 \shortdesc{CompInstList \b{comps};}
 \shortdesc{BindingIndic \b{binding};}
 \shortdesc{BlockConfig \b{config};}
\end{ndesc}

\smallskip
\begin{ndesc}{\b{CompInstList} : Node}{}
 \shortdesc{CompInstList \b{link};}
 \shortdesc{ComponentInst \b{inst};}
\end{ndesc}

\smallskip
\shortndesc{\b{Dummy} : Node}{}
\smallskip
\shortndesc{\b{AssociationList} : Dummy : Node}{}

